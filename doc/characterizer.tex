\chapter{Timing Characterizer}
\label{chap:timing_characterizer}

This chapter describes the automatic timing characterizer for full-custom
digital blocks.

It uses the following input data:
\begin{enumerate}
	\item Parasitics net list of the design that has to be characterized
	\item Output file of the Calibre parasitics extractor that contains input
		capacitances of the input pins of the design
	\item A Python data structure that defines which pins have to be
		characterized
\end{enumerate}

\begin{figure}
	\centering
	%\begin{tikzpicture}[
	yes loop/.style={to path={-- node[above]{Yes} ++(#1,0) |- (\tikztotarget)}},
	vh path/.style={to path={|- node[above right]{} (\tikztotarget)}},
	nonterminal/.style={
		% The shape:
		rectangle,
		% The size:
		%minimum size=6mm,
		% The border:
		very thick,
		draw=black!50,
		 % 50% red and 50% black,
		% and that mixed with 50% white
		% The filling:
		top color=white,
		 % a shading that is white at the top...
		bottom color=black!20, % and something else at the bottom
		% shape
		font=\itshape
		},
	terminal/.style={
		% The shape:
		%rectangle,minimum size=6mm,
		rounded corners,%=3mm,
		% The rest
		very thick, draw=black!50,
		top color=white,bottom color=black!20,
		font=\ttfamily
	},
	]

\colorlet{graphbg}{black!20}

\matrix[row sep=3.0mm,column sep =1.5mm]
{
%First row
\node (in1) [nonterminal] {Signal specification data}; &
\node (in2) [nonterminal] {Parasitics net list}; &
\node (in3) [nonterminal] {Parasitic capacitances};\\
& \node (start) [terminal] {Split LUT entries}; & \\
% Second row
\node (nlwr1) [terminal] {Generate Netlist}; &
\node (nlwr2) [] {\ldots}; &
\node (nlwr3) [terminal] {Generate Netlist};\\
\node (usim1) [terminal] {Ultrasim}; &
\node (usim2) [] {\ldots}; &
\node (usim3) [terminal] {Ultrasim};\\
\node (result1) [terminal] {Analyze Waveform}; &
\node (result2) [] {\ldots}; &
\node (result3) [terminal] {Analyze Waveform};\\
\node (repeat1) [terminal] {Repeat?}; &
\node (repeat2) [] {\ldots}; &
\node (repeat3) [terminal] {Repeat?};\\
\node (done1) [terminal] {Extract Results}; &
\node (done2) [] {\ldots}; &
\node (done3) [terminal] {Extract Results};\\
 & \node (merge) [terminal] {Merge}; & \\
 & \node (append) [terminal] {Append input cap.}; & \\
 & \node (write) [terminal] {Write library file}; & \\
};

\graph{
(in1) -> (start); (in2) -> (start); (in3) -> (start);
};
\graph{
(merge) -> (append); (append) -> (write);
};
\foreach \i in {1,3}
	\graph [edge quotes=near start]{
		(start) -> (nlwr\i) -> (usim\i) -> (result\i) ->
		(repeat\i);% -> (done\i);
	};

\graph{ (repeat1) -> [yes loop=-20mm] (nlwr1);
%	(done2) -> [no loop=-12mm](nlwr2);
	(repeat3) -> [yes loop=+20mm](nlwr3);
};
\graph[edge label=No]{
	(repeat1) -> (done1);
	(repeat3) -> (done3);
};

\graph[]{
	(done1) -> [vh path] (merge);
	(done3) -> [vh path] (merge);
};
\begin{pgfonlayer}{background}
\fill[color=graphbg] ($(done3) + (2.2,-0.5)$) rectangle ($(nlwr1) +
(-2.2,0.8)$);
\node[anchor=north west,inner sep=0.2em]
at ($(nlwr1) + (-2.2,0.8)$) {Parallel Threads };
\end{pgfonlayer}

\end{tikzpicture}

	\missingfigure{Figure removed: tikz graphs library and latex distribution version mismatch.}
	\caption{Figure taken from Hartel, 2015, PhD thesis}
	\label{fig:workflow_characterizer}
\end{figure}

The work flow can be seen in \cref{fig:workflow_characterizer}.
The tool will split up the simulations that have to be done to fill a timing
table into separate threads. For example, setup and hold timing for a single
combination of clock and signal rise times and for all pins of the design will
be extracted by one thread. The next thread will simulate the design with a
different pair of clock and signal rise times.

Every thread performs a binary search for the setup and hold timing by varying
the individual time differences of the input signals and the clock signal. This
can be seen in \cref{fig:binary_search_characterizer}.

\begin{figure}
	\centering
	\begin{tikztimingtable}[timing/slope=0.1,
		%timing/name/.style={font=\ttfamily,align=center},
		timing/coldist=2pt,xscale=6.0,yscale=1,
		]
		{clock sig.} & L N(CK) 1.0H\\
		{input \#0} & 0.5L N(n0) 1.5H\\
		{output \#0} & 1.25L 0.75H\\
		{input \#1} & 0.75L N(n1) 1.25H\\
		{output \#1} & 1.25L 0.75H\\
		{input \#2} & 1.0L N(n2) 1.0H\\
		{output \#2} & 1.25L 0.75H\\
		{input \#3} & 1.25L N(n3) 0.75H\\
		{output \#3} & 1.25L 0.5M [black] 0.25L\\
		{input \#4} & 1.125L N(n4) 0.875H\\
		{output \#4} & 1.25L 0.5M [black] 0.25L\\
		{input \#5} & 1.0625L N(n5) 1H\\
		{output \#5} & 1.25L 0.75H\\
	\extracode
	\makeatletter
	\begin{pgfonlayer}{background}
		\begin{scope}[ gray , semithick ]
			\horlines {1,3,5,7,9,11,13}
			\begin{scope}[dashed]
                        \vertlines{1.05}
			\end{scope}

                        \node [ anchor=south west , inner sep=0pt]
                        at ($(row2.east) + (0.1,0)$) {\small
                          $\Delta t = \SI{0.5}{\nano\second}$ };

					    \node [anchor=south west , inner sep=0pt]
					    at ($(row3.east) + (0.1,0)$) {\small $\downarrow$};

                        \node [ anchor = south west , inner sep=0pt]
                        at ($(row4.east) + (0.1,0)$) {\small
                          $\Delta t = \SI{0.25}{\nano\second}$ };

					    \node [anchor=south west , inner sep=0pt]
					    at ($(row5.east) + (0.1,0)$) {\small $\downarrow$};

                        \node [ anchor = south west , inner sep=0pt ]
                        at ($(row6.east) + (0.1,0)$) {\small
                          $\Delta t = \SI{0.0}{\nano\second}$ };

					    \node [anchor=south west , inner sep=0pt]
					    at ($(row7.east) + (0.1,0)$) {\small $\downarrow$};

                        \node [ anchor = south west , inner sep=0pt ]
                        at ($(row8.east) + (0.1,0)$) {\small
                          $\Delta t = \SI{-0.25}{\nano\second}$ };

					    \node [anchor=south west , inner sep=0pt]
					    at ($(row9.east) + (0.1,0)$) {\small $\uparrow$};

                        \node [ anchor = south west , inner sep=0pt ]
                        at ($(row10.east) + (0.1,0)$) {\small
                          $\Delta t = \SI{-0.125}{\nano\second}$ };

					    \node [anchor=south west , inner sep=0pt]
					    at ($(row11.east) + (0.1,0)$) {\small $\uparrow$};

                        \node [ anchor = south west , inner sep=0pt ]
                        at ($(row12.east) + (0.1,0)$) {\small
                          $\Delta t = \SI{-0.0625}{\nano\second}$ };

					    \node [anchor=south west , inner sep=0pt]
					    at ($(row13.east) + (0.1,0)$) {\small $\surd$};

		\end{scope}
	\end{pgfonlayer}
\end{tikztimingtable}%

	\caption{Figure taken from Hartel, 2015, PhD thesis}
	\label{fig:binary_search_characterizer}
\end{figure}

An example \tf{wscript} for using the timing characterizer can be seen in
\cref{sec:recipe_characterizer}.
