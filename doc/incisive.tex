\chapter{Simulations with Cadence Incisive}
\label{chap:incisive}
\section{Incisive HDL simulation workflow}
\marginnote{3-step process}
The Cadence Incisive tool chain implements a three-step workflow for building
\gls{HDL}/mixed-signal simulations. The tool chain consists of four tools (of
which two implement the same of the three steps but for different languages):
\begin{itemize}
	\item \tf{ncvlog}/\tf{ncvhdl} for compiling design files
	\item \tf{ncelab} for elaborating a design hierarchy
	\item \tf{ncsim} for simulating an elaborated design hierarchy
\end{itemize}

These tools will be described in detail in the following subsecitons.

\subsection{ncvlog}
The Verilog compiler \tf{ncvlog} can be invoked the following way:
\begin{lstbashplain}
 ncvlog [options] filename { filename }
\end{lstbashplain}
For the Verilog family of \glspl{HDL}, there are three supported sub-types by
\tf{ncvlog}:
\begin{itemize}
	\item Verilog 2001 can be compiled by invoking \tf{ncvlog} without any
		particular option, just by giving a filename as an argument.
	\item For SystemVerilog, the command line option \tf{-sv} needs to be
		added.
	\item For Verilog-AMS, the command line option \tf{-ams} needs to be
		added.
\end{itemize}

\marginnote{worklib}
All code is compiled into design libraries having \ti{Cadence}'s proprietary
format. The library into which the files are compiled is called the working
library. This library can be specified by the user. If the user does not
specify the working library, then, by default, \tf{ncvlog} creates a
sub-directory called \ti{INCA\_libs} in the current working directory. This
subdirectory contains a \gls{cds} called \ti{worklib}, which is defined as the
default working library and contains the actual binary file with all compiled
code.

This default behavior can be change by adding two files to the picture. These
are called \ti{cds.lib} and \ti{hdl.var}.

\marginnote{cds.lib}
The \ti{cds.lib} file contains definitions of \glspl{cds}
and aliases for these libraries. All \glspl{cds} that will be used in any
Cadence software tool have to be defined in a \ti{cds.lib} file that has to be
placed in the current working directory.

\marginnote{hdl.var}
The \ti{hdl.var} file can be used to configure the
compilation process, for example by defining which of the existing \glspl{cds}
should be used as the current working library. To change the currently used
working library, the \tf{hdl.var} file has to contain the following code:
\begin{minted}{verilog}
DEFINE WORK ./worklib
\end{minted}

\marginnote{\tf{VIEW\_MAP}}
The \ti{hdl.var} file can also be used to specify the name of the \gls{cds}
view into which the \ti{Verilog} files are compiled. The view names can also be
chosen depending on the suffix of the design files. By default, the view name
\ti{module} is chosen for all verilog files, however it can be modified with
the \ti{VIEW\_MAP} option in the \ti{hdl.var} file.
\begin{minted}{verilog}
DEFINE VIEW_MAP (.vams => behav, \
                  +    => module)
\end{minted}
Here, the + operator represents any other extension. This particular
\ti{VIEW\_MAP} is used by default in \ti{brICk} when working with the
\ti{cadence\_ius} module.

\marginnote{\tf{LIB\_MAP}}
Another property that is important for the elaboration process (see
\cref{sec:ncelab}) and that can be specified in the \ti{hdl.var} file is the
\ti{LIB\_MAP} property. It defines a mapping of source files (or source file
directories) to \glspl{cds}. For example:
\begin{minted}{verilog}
DEFINE LIB_MAP ( \
        ./design/lib1/... => lib1, \
        ./source => lib2, \
         top.v            => lib3, \
         +                => worklib )
\end{minted}

As stated in the Cadence User Guide \citep{cds2014compverilog}, in this example:
\begin{itemize}
    \item All files in ./design/lib1 and in directories below ./design/lib1 are
 mapped to library lib1.
    \item All files in the ./source directory are mapped to library lib2.
    \item The file top.v is mapped to library lib3.
    \item Everything else is mapped to library worklib.
\end{itemize}

The \ti{LIB\_MAP} property also defines which libraries should be scanned by
\tf{ncelab} during the binding process and in which order these should be
scanned.

\marginnote{include directories}
If any of the Verilog files that have to be compiled for a design use the
\vkey{include} statement, the compiler has to know where to look for these
files. By default, it will only check the current working directory. Other
include directories can be specified with the following command.
\begin{lstbashplain}
 ncvlog -incdir directory [options] filename { filename }
\end{lstbashplain}
Multiple \tf{-incdir} options are legal.

\subsection{ncelab}
The design hierarchy elaboration (or just elaboration) process is implemented
with the tool \tf{ncelab} which can be invoked the following way:
\begin{lstbashplain}
 ncelab [options] [Lib.]Cell[:View] { [Lib.]Cell[:View] }
\end{lstbashplain}

\marginnote{elaboration}
During elaboration, \tf{ncelab} checks whether all instantiated modules (or
interfaces/packages etc.) have previously been compiled into any of the present
\glspl{cds}. It also checks whether port connections match and, in case of
mixed-signal simulations, it automatically inserts interface modules
(translating digital to analog signals and vice-versa) or inserts spice models. 

The top-level module of the design hierarchy has to be specified by giving the
module name in the Cadence-specific Library-Cell-View syntax. If a Verilog
module called \ti{test} has been compiled into a library called \ti{worklib}.
It can be specified like this: \tf{worklib.test:module}, or just
\tf{worklib.test} (if there is only one view for this cell), or even simpler:
\tf{test} (if there is only one library and one view).

\marginnote{The binding process}
Starting with the toplevel module, \tf{ncelab} tries to find compiled modules
for all instances in the design. This process is called \ti{binding}. Details
about the binding process can be found in chapter 14 of \citep{cds2014amssimug}
and the elaboration manual \citep{cds2014ncelab}.

\marginnote{default binding process}
This paragraph gives a short overview of the default binding process and how it
can be influenced:
\begin{enumerate}
    \item If a module has already been bound, the same binding will be used
 again.
    \item The libraries given in the \ti{LIB\_MAP} property are scanned for the
 given module name, taking into account only those view names that are
 specified in the \ti{VIEW\_MAP} property. The libraries and views are scanned
 in the order in which they appear in the \ti{hdl.var} file.
    \item If no binding is found in the libraries specified by \ti{LIB\_MAP},
 check the library of the parent module's binding.
    \item If no binding exists, search all known libraries (as defined by the
 \ti{cds.lib} file.
\end{enumerate}

This default binding process can be overridden by the user using the following
means:
\begin{itemize}
    \item A \ti{Library Map File}
 \begin{minted}{verilog}
library rtlLib "top.v";
library aLib "adder.v";
library gateLib "adder.vg";
\end{minted}
    \item The \vkey{uselib} compiler directive inside the \ti{Verilog} code
 \begin{minted}{verilog}
// File: source/top.v
module top ();
    // You want to use the rtl view in the library called source
    `uselib lib=source view=rtl
    foo a();
    `uselib
    foo b();
    bar c();
endmodule
\end{minted}
    \item The \tf{-binding} option of \tf{ncelab}
 \begin{lstbashplain}
 ncelab [options] -binding [Lib.]Cell[:View] [Lib.]Cell[:View]
\end{lstbashplain}
    \item A Verilog configuration
 \begin{minted}{verilog}
config cfg;
    design rtlLib.top;
    default liblist aLib rtlLib gateLib;
    instance top.a2 liblist gateLib;
    cell foo use gateLib.foo;
endconfig
\end{minted}
    \item A Cadence hierarchy configuration\\
 If you specify as the top-level unit a Cadence Hierarchy view (typically
 called a \ti{config} view), \tf{ncelab} will follow the rules specified in
 this view.
\end{itemize}

\clearpage
\subsection{ncsim}

A simulation run has to be started with the \tf{ncsim} command in the following
way:
\begin{lstbashplain}
 ncsim [options] [Lib.]Cell[:View]
\end{lstbashplain}

If \tf{ncsim} should be started with the \ti{Simvision} GUI, the option
\tf{-gui} has to be passed to \tf{ncsim}.

