\chapter{Simulations with Cadence Incisive}
\label{chap:incisive}
\section{Incisive HDL simulation workflow}
\marginnote{3-step process}
The Cadence Incisive tool chain implements a three-step workflow for building
\gls{HDL}/mixed-signal simulations. The tool chain consists of four tools (of
which two implement the same of the three steps but for different languages):
\begin{itemize}
	\item \tf{ncvlog}/\tf{ncvhdl} for compiling design files
	\item \tf{ncelab} for elaborating a design hierarchy
	\item \tf{ncsim} for simulating an elaborated design hierarchy
\end{itemize}

\subsection{ncvlog}
The Verilog compiler \tf{ncvlog} can be invoked the following way:
\begin{lstbashplain}
 ncvlog [options] filename { filename }
\end{lstbashplain}
For the Verilog family of \glspl{HDL}, there are three supported sub-types by
\tf{ncvlog}:
\begin{itemize}
	\item Verilog 2001 can be compiled by invoking \tf{ncvlog} without any
		particular option, just by giving a filename as an argument.
	\item For SystemVerilog, the command line option \tf{-sv} needs to be
		added.
	\item For Verilog-AMS, the command line option \tf{-ams} needs to be
		added.
\end{itemize}

\marginnote{worklib}
All compiled code is compiled into proprietary design libraries. By default,
\tf{ncvlog} creates a sub-directory called \ti{INCA\_libs} in the current
working directory. This subdirectory contains a \gls{cds} called \ti{worklib},
which is defined as the default working library and contains the actual binary
file with all compiled code.

This default behavior can be change by adding two files to the picture. These
are called \ti{cds.lib} and \ti{hdl.var}.

\marginnote{cds.lib}
The \ti{cds.lib} file contains definitions of \glspl{cds}
and aliases for these libraries. All \glspl{cds} that will be used in any
Cadence software tool have to be defined in a \ti{cds.lib} file that has to be
placed in the current working directory.

\marginnote{hdl.var}
The \ti{hdl.var} file can be used to configure the
compilation process, for example by defining which of the existing \glspl{cds}
should be used as the current working library. To change the currently used
working library, the \tf{hdl.var} file has to contain the following code:
\begin{minted}{verilog}
DEFINE WORK ./worklib
\end{minted}

If any of the Verilog files that have to be compiled for a design use the
\vkey{include} statement, the compiler has to know where to look for these
files. By default, it will only check the current working directory. Other
include directories can be specified with the following command.
\begin{lstbashplain}
 ncvlog -incdir directory [options] filename { filename }
\end{lstbashplain}
Multiple \tf{-incdir} options are legal.

\subsection{ncelab}
The design hierarchy elaboration (or just elaboration) process is implemented
with the tool \tf{ncelab} which can be invoked the following way:
\begin{lstbashplain}
 ncelab [options] [Lib.]Cell[:View] { [Lib.]Cell[:View] }
\end{lstbashplain}

\marginnote{elaboration}
During elaboration, \tf{ncelab} checks whether all instantiated modules (or
interface/packages etc.) have previously been compiled into any of the present
\glspl{cds}. It also checks whether port connections match and, in case of
mixed-signal simulations, it automatically inserts interface modules
(translating digital to analog signals and vice-versa) or inserts spice models. 

The top-level module of the design hierarchy has to be specified by giving the
module name in the Cadence-specific Library-Cell-View syntax. If a Verilog
module called \ti{test} has been compiled into a library called \ti{worklib}.
It can be specified like this: \tf{worklib.test:module}, or just
\tf{worklib.test} (if there is only one view for this cell), or even simpler:
\tf{test} (if there is only one library and one view).
The view name \ti{module} is chosen by default for verilog files, however it
can be modified with the \ti{VIEW\_MAP} option in the \ti{hdl.var} file.
\begin{minted}{verilog}
DEFINE VIEW_MAP (.v    => behav, \
                 .rtl  => rtl, \
                 .gate => gate, \
                  +     => module)
\end{minted}
Here, the + operator represents any other extension.

\subsection{ncsim}

A simulation run has to be started with the \tf{ncsim} command in the following
way:
\begin{lstbashplain}
 ncsim [options] [Lib.]Cell[:View]
\end{lstbashplain}

If \tf{ncsim} should be started with the \ti{Simvision} GUI, the option
\tf{-gui} has to be passed to \tf{ncsim}.

