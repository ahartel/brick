\chapter{Simulation Recipes}
To run and test the examples in this chapter, you can clone the \ti{brICk}
repository\footnote{git clone git@gitviz.kip.uni-heidelberg.de/brick.git} and
go to the path that is specified for each example.

Prior to running any example you need to check that all \ti{Cadence Incisive}
programs are accessible and you need to add the tool paths to your PYTHONPATH
environment variable:
\begin{lstbashplain}
export PYTHONPATH=$PYTHONPATH:/your_brick_path/source/waf:\
/your_brick_path/source/python
\end{lstbashplain}

\section{Cadence Incisive}
For background information about simulating \gls{HDL} design with \ti{Cadence
Incisive} please see \cref{chap:incisive}.
\subsection{Basic RTL simulation}
The most basic \tf{wscript} for RTL simulation with \ti{Cadence Incisive} is
shown in the following listing. A running example of it can be found in the
folder \tf{test/cadence\_ius/sim\_rtl\_only\_0}.
\begin{lstwscript}
def configure(cfg):
    cfg.load('cadence_ius')

def build(bld):
    bld.load('brick_general')
    bld ( features = 'cds_write_libs' )
 
    bld (
        name = 'compile_top',
        features = 'cds_compile_hdl',
        source = bld.convert_string_paths(
            [
                source_file0,
                source_file1,
            ]),
        verilog_search_paths = bld.convert_string_paths(
            [
                search_path0,
                search_path1,
            ]
        ),
    )
 
    bld.add_group()
 
    bld (
        toplevel = 'worklib.tb_top',
        features = 'cds_elab',
        always = True
    )

def run(bld):
    bld (
        features = 'ncsim',
        toplevel = 'worklib.tb_top',
    )

from waflib.Build import BuildContext
class one(BuildContext):
    cmd = 'run'
    fun = 'run'
}
\end{lstwscript}

This \tf{wscript} sets up calls to \tf{ncvlog} for all files given to the
\ti{compile\_top} task generator. Afterwards it runs \tf{ncelab} on the given
toplevel unit. When typing \lstinline[style=BashInputStyle]'./waf run',
\tf{ncsim} is invoked. By default, this starts a GUI version of \tf{ncsim}.
\subsubsection{Disabling the \tf{ncsim} GUI}
You can also start \tf{ncsim} without a GUI, by modifying the \tf{configure}
function of the wscript like this:
\begin{lstwscript}
def configure(cfg):
    cfg.load('cadence_ius')
    cfg.env['NCSIM_OPTIONS'].remove('-gui')
\end{lstwscript}

\subsubsection{Changing the worklib's name}
Any working library that is explicitely given by the user (i.e. written into
the \tf{hdl.var} file has to be listed in the \tf{cds.lib} file as well and the
directory that contains the library has to be created by the user.

To tell the \ti{Incisive} tools to use a different working library the
\tf{wscript} has to be modified like this:
\begin{lstwscript}
def configure(conf):
    cfg.load('cadence_ius')
    cfg.env.CDS_LIBS = ['./myworklib']
    cfg.env.CDS_WORKLIB = 'myworklib'
def build(bld):
    ...
    bld (
        toplevel = 'mywork.tb_top',
        features = 'cds_elab',
        always = True
    )

def run(bld):
    bld (
        features = 'ncsim',
        toplevel = 'worklib.tb_top',
    )

...
\end{lstwscript}
This example code can be found in the folder
\tf{test/cadence\_ius/sim\_rtl\_only\_0} of the repository.

\subsection{Schematic-Based Mixed-signal RTL simulation}

\subsection{Layout-Based Mixed-signal RTL simulation}

\section{Modelsim}
\subsection{Basic RTL simulation}

\begin{lstwscript}
def configure(conf):
    conf.load('brick_general')
    conf.load('modelsim')

def build(bld):
    bld.load('brick_general')

    bld (
        name = 'compile_top',
        source = bld.convert_string_paths(
            [   
                'tb.sv',
                'top.sv',
            ])  
            ,   
        features = 'modelsim',
    )   


def run(bld):
    bld (
        features = 'vsim',
        toplevel = 'worklib.tb',
    )   

from waflib.Build import BuildContext
class one(BuildContext):
    cmd = 'run'
    fun = 'run'
\end{lstwscript}


