\chapter{Recipes}
\label{chap:recipes}
To run and test the examples in this chapter, you can go to to the path that is
specified for each example.

Prior to running any example you need to check that all \ti{Cadence Incisive}
programs are accessible and you need to follow the instructions in
\cref{sec:install}.

\section{Simulating with Cadence Incisive}
For background information about simulating \gls{HDL} design with \ti{Cadence
Incisive} please see \cref{chap:incisive}.
\subsection{Basic RTL simulation}
\label{sec:ius_basic_rtl}
The most basic \tf{wscript} for RTL simulation with \ti{Cadence Incisive} is
shown in the following listing. A running example of it can be found in the
folder \tf{test/cadence\_ius/sim\_rtl\_only\_0}.
\begin{lstwscript}
def configure(cfg):
    cfg.load('cadence_ius')

def build(bld):
    bld ( features = 'cds_write_libs' )
 
    bld (
        name = 'compile_top',
        features = 'cds_compile_hdl',
        source = bld.convert_string_paths(
            [
                'source_file0',
                'source_file1',
            ]),
        verilog_search_paths = bld.convert_string_paths(
            [
                'search_path0',
                'search_path1',
            ]
        ),
    )
 
    bld.add_group()
 
    bld (
        toplevel = 'worklib.tb_top',
        features = 'cds_elab',
        always = True
    )

def run(bld):
    bld (
        features = 'ncsim',
        toplevel = 'worklib.tb_top',
    )

from waflib.Build import BuildContext
class one(BuildContext):
    cmd = 'run'
    fun = 'run'
}
\end{lstwscript}

This \tf{wscript} sets up calls to \tf{ncvlog} for all files given to the
\ti{compile\_top} task generator. Afterwards it runs \tf{ncelab} on the given
toplevel unit. When typing \mint{bash}'./waf run',
\tf{ncsim} is invoked. By default, this starts a GUI version of \tf{ncsim}.
\subsubsection{Disabling the \tf{ncsim} GUI}
You can also start \tf{ncsim} without a GUI, by modifying the \tf{configure}
function of the wscript like this:
\begin{lstwscript}
def configure(cfg):
    cfg.load('cadence_ius')
    cfg.env['NCSIM_OPTIONS'].remove('-gui')
\end{lstwscript}

\subsubsection{Changing the worklib's name}
Any working library that is explicitely given by the user (i.e. written into
the \tf{hdl.var} file has to be listed in the \tf{cds.lib} file as well and the
directory that contains the library has to be created by the user.

To tell the \ti{Incisive} tools to use a different working library the
\tf{wscript} has to be modified like in the following code example. A running
example of it can be found in the folder
\tf{test/cadence\_ius/sim\_rtl\_only\_1}.

\begin{lstwscript}
def configure(conf):
    cfg.load('cadence_ius')
    cfg.env.CDS_LIBS = ['./myworklib']
    cfg.env.CDS_WORKLIB = 'myworklib'

def build(bld):
    ...
    bld (
        toplevel = 'mywork.tb_top',
        features = 'cds_elab',
        always = True
    )

def run(bld):
    bld (
        features = 'ncsim',
        toplevel = 'worklib.tb_top',
    )

...
\end{lstwscript}

It is worth noting that the workflow presented in this subsection also has to
be used for \gls{RVM}.

\subsubsection{Changing the view mapping}
\todo[inline]{Enter view mapping snippet}
\subsubsection{Using multiple snapshots}
\todo[inline]{Enter multiple snapshot snippet}

\clearpage
\subsection{Behavioral-Based Mixed-signal simulation}
\label{sec:ius_behave_rtl}
If you want to use \ti{Verilog-AMS} behaioral models that contain analog blocks
or signals of discipline electrical, the following \tf{wscript} snippet can
help you to set up a simulation. This \tf{wscript} can also be used when you
want to simulate schematics and the \ti{Verilog-AMS} files that are derived
from these schematics have already been netlisted. If, on the other hand, your
schematics have not yet been netlistet and you want them to get netlisted
automatically, please refer to \cref{sec:ius_schematic_rtl}.

The example code presented in this section can be found in the folder\\
\tf{test/cadence\_ius/sim\_mixed\_signal\_0} of the repository.

\begin{lstwscript}
import os

def configure(conf):
    conf.env.CDS_MIXED_SIGNAL = True

    conf.load('cadence_ius')

    conf.env['NCELAB_OPTIONS'].extend([
        '-amsconnrules', 'ConnRules_12V_full_fast',
        'ConnRules_12V_full_fast',
    ])

def build(bld):
    bld ( features = 'cds_write_libs' )

    bld (
        name = 'compile_top',
        features = 'cds_compile_hdl',
        source = bld.convert_string_paths(
            [
                'source_file0',
                'source_file1',
                os.environ['BRICK_DIR']+'/source/verilog-ams/ConnRules12.vams',
            ]),
        verilog_search_paths = bld.convert_string_paths(
            [
                'search_path0',
                'search_path1',
            ]
        ),
    )
    bld.add_group()
    bld (
        toplevel = 'worklib.tb',
        features = 'cds_elab',
        always = True
    )   

def run(bld):
    bld (
        features = 'ncsim',
        toplevel = 'worklib.tb',
        stop_time = '100n',
    )   

from waflib.Build import BuildContext
class one(BuildContext):
    cmd = 'run'
    fun = 'run'
\end{lstwscript}

In this listing, compared to the listing in \cref{sec:ius_basic_rtl}, lines 2,
6-9, 22 and 42 have been added and mark the important changes. The rest of the
listing has remained unchanged. The additional lines' purposes are described
next:
\begin{description}
    \item[Line 2:] The environment variable \tf{CDS\_MIXED\_SIGNAL} enables the
 mixed-signal flow. In particular, it changes the default options that are
 passed to \tf{ncelab} and tells to the task generator for the \tf{ncsim} task
 to create and use an analog control file.
    \item[Lines 6-9:] These add options to the \tf{ncelab} call that tell it to
 use the connect rules that are defined in line 22.
    \item[Line 22:] Adds a Verilog-AMS source file to the compilation process.
 This file defines connect rules for the automatic connect module insertion
 process that is handled by \tf{ncelab}.
    \item[Line 42:] Tells the task generator for the \tf{ncsim} task to set the
 stop time of the transient simulation to 100 ns.
\end{description}

\clearpage
\subsection{Schematic-Based Mixed-signal simulation}
\label{sec:ius_schematic_rtl}

The example code presented in this section can be found in the folder\\
\tf{test/cadence\_ius/sim\_mixed\_signal\_1} of the repository.

\begin{lstwscript}
import os

def configure(conf):
    conf.env.CDS_MIXED_SIGNAL = True

    conf.load('cadence_ius')
    conf.load('cadence_mixed_signal')
                                                                                       
    conf.env['CDS_LIBS']['brick_test'] = '../../cdslib/'
    conf.env.CDS_LIB_INCLUDES = [ 
        '$TSMC_DIR/oa/cds.lib',
    ]   

    conf.env['NCELAB_OPTIONS'].extend([
        '-amsconnrules', 'ConnRules_12V_full_fast', 'ConnRules_12V_full_fast',
        '-libverbose',
        '-modelpath',
        os.environ['TSMC_DIR']+'/oa/models/spectre/toplevel.scs(tt_lib)',
    ])

def build(bld):
    bld ( features = 'cds_write_libs' )

    bld (
        feature = 'cds_mixed_signal',
        cellview = 'lib.cell:view',
    )

    bld (
        name = 'compile_top',
        source = bld.convert_string_paths(
            [
                'source_file0',
                'source_file1',
                os.environ['BRICK_DIR']+'/source/verilog-ams/ConnRules12.vams',
            ]),
        features = 'cds_compile_hdl',
        verilog_search_paths = bld.convert_string_paths(
            [
                'search_path0',
                'search_path1',
            ]
    )
    bld.add_group()
    bld (
        toplevel = 'worklib.tb',
        features = 'cds_elab',
        always = True
    )

def run(bld):
    bld (
        features = 'ncsim',
        toplevel = 'worklib.tb',
        stop_time = '100n',
    )

from waflib.Build import BuildContext
class one(BuildContext):
    cmd = 'run'
    fun = 'run'
\end{lstwscript}

In this listing, compared to the listing in \cref{sec:ius_behave_rtl}, lines
25-28 contain the important changes. Otherwise the snippet has remained
unchanged (except for the implicit changes in the anonymous source file
placeholders).
\begin{description}
    \item[Line 42:] Adds a task generator call that creates 2 tasks, one for
        a netlisting task and a second one for a compilation task that compiles
        the resulting netlist.
\end{description}

\warningsign Note that the \ti{cadence\_mixed\_signal} tool needs a cell view
of type \ti{config}. If you don't already have a \ti{config} cell view at hand,
you can let \ti{brICk} create one by adding the following lines to your
\tf{wscript}.
\begin{lstwscript}
bld (                                                                          
    features = 'cds_config',
    name = 'create_some_config',
    libs = ['lib','tsmcN65'],
    cellview = 'lib.cell:config',
    update_outputs = True
)
\end{lstwscript}

Here, the \tf{cellview} property defines the toplevel cell view of the
hierarchy. The output cell view (the one that will be created by \ti{brICk})
will, by default, have the same cell name as the source cell view but will
reside in the working library and have view name \tf{brick\_config}. To
override this behavior, you can add the property \tf{config\_cellview} to the
call. It has to contain a string of type \tf{lib.cell:view}.

The \tf{update\_outputs} property has to be set to \ti{True}, if the file
that will be generated by this task does not reside in the build directory.

\subsubsection{Explicitely changing view names of compilation units}
In some cases, there are multiple representations of a module, e.g. a
behavioral and a full-custom schematic. It can happen that both are given as
\ti{Verilog-AMS} code (one written by the user and the other one generated by
the netlister. Therefore, both files would be compiled into the same view, one
overwriting the other. To differentiate between both you can compile both into
different views. This can be done with the \tf{view} attribute of the task
generator, as in the following example.

Once you have compiled both representations of the same module into different
views, you can tell \tf{ncelab} which one to pick by using the \tf{binding}
attribute of the task generator.

The example code presented in this subsection can be found in the folder\\
\tf{test/cadence\_ius/sim\_mixed\_signal\_2} of the repository.

\begin{lstwscript}
...

def build(bld):
    bld ( features = 'cds_write_libs' )

    bld (
        feature = 'cds_mixed_signal',
        cellview = 'lib.cell:view',
        view = 'schematic',
    )

    bld (
        name = 'compile_top',
        view = 'behavioral',
        source = bld.convert_string_paths(
            [
                'source_file0',
                'source_file1',
                os.environ['BRICK_DIR']+'/source/verilog-ams/ConnRules12.vams',
            ]),
        features = 'cds_compile_hdl',
        verilog_search_paths = bld.convert_string_paths(
            [
                'search_path0',
                'search_path1',
            ]
    )
    bld.add_group()
    bld (
        toplevel = 'worklib.tb',
        features = 'cds_elab',
        always = True,
        binding = ['lib.cell:schematic'],
    )

...
\end{lstwscript}

When using this attribute, \ti{brICk} translates it to a \tf{-view} option of
the \tf{ncvlog} calls.

\subsection{Layout-Based Mixed-signal simulation}

\section{Simulating with Modelsim}
\subsection{Basic RTL simulation}
\label{sec:modelsim_basic_rtl}

\begin{lstwscript}
def configure(conf):
    conf.load('modelsim')

def build(bld):

    bld (
        name = 'compile_top',
        features = 'modelsim',
		source = bld.convert_string_paths(
            [
                'source_file0',
                'source_file1',
            ]),
        verilog_search_paths = bld.convert_string_paths(
            [
                'search_path0',
                'search_path1',
            ]
        ),
    )   


def run(bld):
    bld (
        features = 'vsim',
        toplevel = 'worklib.tb',
    )   

from waflib.Build import BuildContext
class one(BuildContext):
    cmd = 'run'
    fun = 'run'
\end{lstwscript}

\subsubsection{Changing the worklib's name}
\begin{lstwscript}
def configure(conf):
    conf.load('modelsim')
    conf.env.MODELSIM_WORKLIBS.append('mywork')

def build(bld):

    bld (
        name = 'compile_top',
        features = 'modelsim',
        worklib = 'mywork',
        ...
    )   


def run(bld):
    bld (
        features = 'vsim',
        toplevel = 'mywork.tb',
    )   

...
\end{lstwscript}
In this listing, compared to the listing in \cref{sec:modelsim_basic_rtl},
lines 3, 10 and 18 have been added and mark the important changes. The rest of
the listing has remained unchanged. The additional lines' purposes are
described next:
\begin{description}
    \item[Line 3:] Add the library to the global list of working libraries.
		This is necessary to have the libraries available when running the
		simulation.
    \item[Line 10:] Tell the task generator to compile its source files to the
		working library \ti{mywork}
	\item[Line 18:] Select the design unit tb from working library \ti{mywork}
		as the design top level.
\end{description}

\section{Checking and Extracting Layouts with Calibre}
\label{sec:calibre_recipes}
\subsection{Design Rule Check}
\label{sec:calibre_recipes_drc}
If you already have a \ti{GDSII} file, you can use the following minimal
\tf{wscript} to set up a \gls{DRC} with \ti{Calibre}.
This example can be found in the folder\\
\tf{test/calibre/drc\_inverter}.
\begin{lstwscript}
def configure(conf):

    conf.load('calibre_drc')

def build(bld):

    bld (
        name = 'calibre_drc_inverter',
        features = 'calibre_drc',
        cellname = 'inverter',
        layout_gds = bld.find_node('your.gds'),
        includes = [
            bld.root.find_node(/your/process/rule/file))
        ],
        unselect_checks = [
            #'M9.DN.2L',
        ]
    )

\end{lstwscript}

The \ti{task generator} called \tf{calibre\_drc\_inverter} sets up a rule file
for you. It will add include statements for the files that you'd specify
on line 11 and it will add \tf{DRC UNSELECT CHECK} statements for the checks that
you'd specified in lines 12-14.

\subsubsection{Running Calibre Designrev to inspect errors}
If you want to run \ti{Calibre Designrev} to inspect the errors that \gls{DRC}
has found in your layout you can add the following code to your \tf{wscript}
and execute \tf{./waf run} after you've run the check.

\begin{lstwscript}
def run(bld):
    bld (
        features = 'calibre_rve_drc',
        cellname = 'inverter',
        gds = bld.path.find_node('../brick_test_inverter.gds'),                                                                                                               
    )   

from waflib.Build import BuildContext
class one(BuildContext):
    cmd = 'run'
    fun = 'run'
\end{lstwscript}

\subsubsection{Generating the GDSII file on-the-fly}
If you want to run a \gls{DRC} on a layout that resides in one of your Cadence
libraries, you can let \ti{brick} generate a streamed-out \ti{GDSII} file for
you on-the-fly.
This example can be found in the folder
\tf{test/calibre/drc\_inverter\_streamout}.
\begin{lstwscript}
def configure(conf):

    conf.env['CDS_LIBS'] = {'brick_test': '../../cdslib/'}
    conf.env.CDS_LIB_INCLUDES = [
        '$TSMC_DIR/oa/cds.lib',
    ]

    conf.load('cadence_strmout')
    conf.load('calibre_drc')

def build(bld):
    bld ( features = 'cds_write_libs' )

    # Additional task generator for streamout
    inverter_streamout = bld (
        name = 'streamout_inverter',
        features = 'cds_strmout',
        cellview = 'brick_test.inverter:layout',
    )

    bld (
        features = 'calibre_drc',
        cellname = 'inverter',
        layout_gds = inverter_streamout.get_cadence_strmout_gds_node(),
        includes = [
            bld.root.find_node(/your/process/rule/file))
        ],
        unselect_checks = [
            #'M9.DN.2L',
        ]
    )
\end{lstwscript}

\subsection{Layout Versus Schematic}
The following code example demonstrates how to perform an \gls{LVS} check on a
layout and schematic that resides in one of your Cadence design libraries.
If you intend to perform the check on existing files (GDSII and spice netlist)
you can do so by changing lines 36 and 37 and by removing the netlisting and
streamout task generators.

The following example can be found in the folder
\tf{test/calibre/lvs\_inverter}.

\begin{lstwscript}
import os

def configure(conf):

    conf.env['CDS_LIBS'] = {'brick_test': '../../cdslib/'}
    conf.env.CDS_LIB_INCLUDES = [
        '$TSMC_DIR/oa/cds.lib',
    ]

    conf.load('cadence_strmout')
    conf.load('cadence_netlist')
    conf.load('calibre_lvs')

def build(bld):

    bld ( features = 'cds_write_libs' )

    inverter_streamout = bld (
        name = 'streamout_inverter',
        features = 'cds_strmout',
        cellview = 'brick_test.inverter:layout',
    )

    inverter_netlist = bld (
        name = 'cds_netlist_lvs_inverter',
        features = 'cds_netlist_lvs',
        cellview = 'brick_test.inverter:schematic',
        include = '/some/netlist/you/want/to/include'
    )


    bld (
        features = 'calibre_lvs',
        layout_cellname = 'inverter',
        source_cellname = 'inverter',
        layout_gds = inverter_streamout.get_cadence_strmout_gds_node(),
        source_netlist = inverter_netlist.get_cds_netlist_lvs_node(),
        includes = [
            bld.root.find_node(/your/process/rule/file))
        ],
    )

\end{lstwscript}

Please note that you can use the \ti{include} attribute of the
\tf{cds\_netlist\_lvs} \ti{task generator} to include a netlist into the result
netlist that is generated from the schematic. 

\subsection{Parasitics Extraction}
\label{sec:calibre_recipes_pex}

This example demonstrates how to perform a parasitics extraction on a layout
that resides in one of your Cadence design libraries. It can be found in\\
\tf{test/calibre/pex\_inverter}.

\begin{lstwscript}
def configure(conf):

    conf.load('calibre_pex')

def build(bld):

    pex = bld (
        name = 'xrc_inverter',
        features = 'calibre_pex',
        cellname = 'inverter',
        layout_gds = bld.path.find_node('your_gdsii_file'),
        includes = [
            bld.root.find_node(/your/process/rule/file))
        ],
        extract_include_recursive = True,
        only_extract_nets = [
            # Some net names
        ],
        # Some additional statements you want to put into the rule file
        mixins = [
            'LAYOUT RENAME TEXT "/</\\[/g" "/>/\\]/g"',
            'SOURCE RENAME TEXT "/</\\[/g" "/>/\\]/g"',
        ],
    )
\end{lstwscript}

Note that lines 15-23 are optional and can be used to extract only those nets
that you specify and add several Calibre rule-file statements to the rule file
that is generated by \ti{brICk}.
If you specify nets in the \tf{only\_extract\_nets} attribute, these net names
relate to the layout database. Usually, you would want to specify these nets
names as they appear in the schematic. For the tool to be able to match these
source net names to the layout, an LVS has to have been performed. The next
example shows how to combine the two.

\subsubsection{Using source netlist names}
The following example shows how to perform an \gls{LVS} check before running a
parasitics extraction. This allows you to use source (i.e. schematic) net names
in the \tf{only\_extract\_nets} attribute.
A running example of this can be found in
\tf{test/calibre/pex\_inverter\_sourcenames\_streamout}.

\begin{lstwscript}
def build(bld):
    ...

    # perform an LVS first
    bld (
        features = 'calibre_lvs',
        layout_cellname = 'inverter',
        source_cellname = 'inverter',
        layout_gds = inverter_streamout.get_cadence_strmout_gds_node(),
        source_netlist = inverter_netlist.get_cds_netlist_lvs_node(),
        includes = [
            bld.root.find_node(/your/process/rule/file))
        ],
    )

    pex = bld (
        name = 'xrc_inverter',
        features = 'calibre_pex',
        cellname = 'inverter',
        layout_gds = inverter_streamout.get_cadence_strmout_gds_node(),
        use_sourcenames = True,
        includes = [
            bld.root.find_node(/your/process/rule/file))
        ],
        extract_include_recursive = True,
        only_extract_nets = [
            # schematic net names
        ],
        mixins = [
            'LAYOUT RENAME TEXT "/</\\[/g" "/>/\\]/g"',
            'SOURCE RENAME TEXT "/</\\[/g" "/>/\\]/g"',
        ],
    )
\end{lstwscript}
\subsubsection{Using xcells}
If you want to use xcells to reduce the complexity of the extracted netlist,
you can do so by specifying them via the \tf{hcells} and \tf{xcells} attributes
of the \ti{task generators}.
A running example of this can be found in\\
\tf{test/calibre/pex\_inverter\_sourcenames\_streamout}.

\begin{lstwscript}
def build(bld):
    ...

    # perform an LVS first
    bld (
        features = 'calibre_lvs',
        layout_cellname = 'inverter',
        source_cellname = 'inverter',
        layout_gds = inverter_streamout.get_cadence_strmout_gds_node(),
        source_netlist = inverter_netlist.get_cds_netlist_lvs_node(),
        includes = [
            bld.root.find_node(/your/process/rule/file))
        ],
        hcells = [
            #        'synapse synapse',

        ]
    )

    pex = bld (
        name = 'xrc_inverter',
        features = 'calibre_pex',
        cellname = 'inverter',
        layout_gds = inverter_streamout.get_cadence_strmout_gds_node(),
        source_netlist = inverter_netlist.get_cds_netlist_lvs_node(),
        includes = [
            bld.root.find_node(/your/process/rule/file))
        ],
        xcells = [
            #        'synapse synapse',
        ],
    )
\end{lstwscript}

This example assumes that there is a circuit called synapse that you want to
extract hierarchically.

\section{Timing characterization using brICk}
\label{sec:recipe_characterizer}

Before running the timing characterization you will probably want to run an
extraction with source names as described in \label{sec:calibre_recipes_pex}.
You should use the option\\ \tf{only\_extract\_nets} to extract all nets that are
needed for the characterization. These include obviously the input and output
pins, but also the nets that are connected to the other terminal of the
flip-flops you want to characterize, for example the outputs of all flip-flops
that are connected to input pins of your design.

The most basic \tf{wscript} for characterization of four flip-flop will be described in the following listing. A running example of it can be found in the
folder \tf{test/characterizer/simple\_ff\_test}.

\begin{minted}[tabsize=2,mathescape,linenos,numbersep=5pt,frame=lines,framesep=2mm]{python}
def configure(conf):
	conf.load('brick_general')
	conf.load('brick_characterize')
	# if you want to extract your design first
	conf.load('cadence_base')
	conf.load('cadence_absgen')
	conf.load('cadence_strmout')
	conf.load('cadence_netlist')
	conf.load('calibre_lvs')
	conf.load('calibre_pex')

	conf.env.CELL_NAME_LAYOUT = ('brick_test','sr_data_ffs','layout')
	conf.env.CELL_NAME_SOURCE = ('brick_test','sr_data_ffs','schematic')

def build(bld):
	# Put extraction of layout and schematic here
	# Also extraction has to be defined here
	pex = bld (
		features = 'calibre_pex',
		...
	)	

	bld (
		name = 'char_ffs',
		features = 'brick_characterize',
		lib_name = bld.env.CELL_NAME_SOURCE[0],
		cell_name = bld.env.CELL_NAME_SOURCE[1],

		rise_threshold = 0.5,
		fall_threshold = 0.5,
		default_max_transition = 0.2,

		inputs = [
			'd_in[1:0]',
			'd_out_ff[1:0]',
			],

		inouts = [],

		outputs = [
			'd_in_ff[1:0]',
			'd_out[1:0]',
			],

		static_signals = {
			#'pc_confb[3:0]': 0,
			},

		powers = {
			'vdd': 1.2,
			'gnd': 0,
			},

		clocks = {
			'clk': 'R'
		},

		input_timing_signals = {
			'd_in[1:0]': [
				'clk',
				'd_in_ff[=index=]',
				'positive_unate'
			],
            'd_out_ff[1:0]': [
				'clk',
				'd_out[=index=]',
				'positive_unate'
			],
		},

		output_timing_signals = {
			'd_out[1:0]' : [
				'clk',
				'd_out_ff[=index=]',
				'positive_unate'
			],
			'd_in_ff[1:0]' : [
				'clk',
				'd_in[=index=]',
				'positive_unate'
			],
		},

		# input files
		circuit_netlist_path = \
		pex.get_calibre_pex_output_file_node('.pex.netlist').abspath(),
		model_netlist_path = \
		os.environ['BRICK_DIR']+'/source/spice/include_all_models_tsmc.scs',
		# output files
		output_netlist_file = './netlists/char_data_ffs.sp',
		output_lib_file = bld.bldnode.make_node('./brick_test_data_ffs.lib'),

		# templates for lib file
		constraint_template = [
			# related_pin_tranisition (ns)
			[0.01,0.05,0.5],
			# constrainted_pin_tranisition (ns)
			[0.01,0.05,0.5],
		],

		delay_template = [
			# input_net_transition (ns)
			[0.0049, 0.0125, 0.0277, 0.0582, 0.1192, 0.2412, 0.4851],
			# total_output_net_capacitance;
			[0.00077, 0.0017, 0.00355, 0.00725, 0.01466, 0.02947, 0.0591],
		],

		parasitics_report = \
		pex.get_calibre_pex_output_file_node('.pex.report'),
		logfile = 'some_logfile.log'

		# debugging switch
		only_rewrite_lib_file = False,
		skip_setup_hold = False,
		skip_delays = False,
	)
\end{minted}

The last task generator in this \tf{wscript} defines the timing
characterization.
It demonstrates which information is needed in addition to
the circuit net list to run a characterization. The most interesting piece of
information is given in lines 62-86. In particular, lines 62-73 specify for the
input signals with a timing constraint their clock signal, their related signal
(i.e. the signal that is connected to the output of the flip-flop that will be
measured) and the edge relationship of input and output signals at the
flip-flop. Here, \texttt{positive\_unate} means, that if the input of a
flip-flop sees a legal transition from 0 to 1 at the next clock edge, the
relevant output of the same flip-flop will transition in the same direction.
The term \texttt{negative\_unate} is used for the opposite relationship.

In the same sense, lines 75-86 specify for the output signals with timing
constraints their clock signal, related signal (i.e. the signal that is
connected to the input of the flip-flop) and their transition relationship.

Since both, the input signals and the output signals, are buses the related
signal names will differ for every single bit of the bus. Therefore, as can be
seen in line 61, the related signal \tf{d\_in\_ff} for the input signal
\tf{d\_in} is named using the \tf{=index=} placeholder. This placeholder will
always be expanded to the current index of the bus' signal.
You can also apply some mathematics to this placeholder, this can be seen in
the next subsection.

\subsection{More complicated net names}

This code example also demonstrates the syntax that has been chosen to specify
the related signal for an input or output signal conveniently. It has to be
kept in mind that the circuit that will be simulated has been automatically
generated off a layout description file and therefore contains net names that
can be lengthy and hardly readable. These net names usually emerge when
instance names across several layers of hierarchy are concatenated (as in the
example). Therefore, mathematical operations can be specified to calculate the
net names from the index of the currently investigated input signal. For
example \texttt{=index/8=} evaluates to the index of the current input signal
(in the bus) divided by 8.

\begin{lstwscript}
clocks = { 'clk': 'R' }
input_timing_signals = {
	'write': [
		'clk',
		'N_XI1_write_int_XI1_MM123_g',
		'positive_unate'
	],
	...
	'd_in[0:511]' : [
		'clk',
		'XI12[=index/8=]_Xdriver_pre[=index%8=]_net047',
		'positive_unate'
	],
}

output_timing_signals = {
	'd_out[0:511]' : [
		'clk',
		'XI12[=index/8=]_Xdriver_pre[=7-index%8=]_out',
		'positive_unate'
	],
}
\end{lstwscript}


