\chapter{Simulation Recipes}
\section{Cadence Incisive}
\subsection{Basic RTL simulation \wscript}
The most basic \wscript should look like this:
\lstwscript
	def configure(cfg):
        cfg.load('brick_general')
        cfg.load('cadence_base')
        cfg.load('cadence_ius')
 
    def build(bld):
        bld.load('brick_general')
        bld ( features = 'cds_write_libs' )
     
        bld (
            name = 'compile_top',
            features = 'cds_compile_hdl',
            source = bld.convert_string_paths(
                [
                    source_file0,
                    source_file1,
                ]),
            verilog_search_paths = bld.convert_string_paths(
                [
                    search_path0,
                    search_path1,
                ]
            ),
        )
     
        bld.add_group()
     
        bld (
            toplevel = 'worklib.tb_top',
            features = 'cds_elab',
            always = True
        )
 
    def run(bld):
        bld (
            features = 'ncsim',
            toplevel = 'worklib.tb_top',
        )
 
    from waflib.Build import BuildContext
    class one(BuildContext):
        cmd = 'run'
        fun = 'run'

\end{minted}

This \wscript sets up calls to \ncvlog for each file given to the \enquote{compile\_top} task generator.
Afterwards it runs ncelab on the given toplevel unit.
When typing \lstinline[style=BashInputStyle]'./waf run', \ncsim is invoked. By default, this starts a GUI version of \ncsim.
\subsection{Disabling the \ncsim GUI}
You can modify this behavior by adding the following line to the end of the configure step.
\lstwscript
def configure(cfg):
	...
	cfg.env['NCSIM_OPTIONS'].remove('-gui')
\end{minted}

\subsection{Changing the worklib's name}

\todo{Make a cake}\missingfigure{Hallo}

\subsection{Schematic-Based Mixed-signal RTL simulation \wscript}

\subsection{Layout-Based Mixed-signal RTL simulation \wscript}

\section{Modelsim}
\subsection{Basic RTL simulation \wscript}

\lstwscript
def configure(conf):
    conf.load('brick_general')
    conf.load('modelsim')

def build(bld):
    bld.load('brick_general')

    bld (
        name = 'compile_top',
        source = bld.convert_string_paths(
            [   
                'tb.sv',
                'top.sv',
            ])  
            ,   
        features = 'modelsim',
    )   


def run(bld):
    bld (
        features = 'vsim',
        toplevel = 'worklib.tb',
    )   

from waflib.Build import BuildContext
class one(BuildContext):
    cmd = 'run'
    fun = 'run'
\end{minted}


