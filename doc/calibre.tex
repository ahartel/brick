\chapter{Checking and Extracting Layouts with Calibre}
\label{chap:calibre}

\ti{Calibre} is a tool by \ti{Mentor Graphics} that allows to run several
checks on \gls{ASIC} design files. These checks include:
\begin{itemize}
	\item \acrfull{DRC}
		checks a layout design file for violations of the design rules that are
		specified by the manufacturer.
	\item \acrfull{LVS}
		checks a layout design file for correctness according to a so calles
		source netlist (usually a schematic) of the circuit that has been drawn
		in this layout.
	\item \acrfull{PEX}
		extracts the parasitic passive elements from a circuit's layout and
		generates a \ti{Spice} netlist which contains the explicitely present
		elements of the circuit plus the parasitic passive elements.
\end{itemize}

All of these features can be invoked via the same tool, called \ti{Calibre}.
Furthermore, all of these tools use so called \acrfull{SVRF} files to specify
most of the options for the checks and to specify the input files.

\begin{table}[h]
	\begin{center}
\begin{tabular}{l|l}
Layout System & The format of the layout data.\\
Layout Path & The location of the layout data.\\
Layout Primary & The top-level cell within the layout data.\\
DRC Results Database & The results database pathname and format.\\
\hline
Layout System & The format of the layout data.\\
Layout Path & The location of the layout data.\\
Layout Primary & The top-level cell within the layout data.\\
Source System & The format of the source data.\\
Source Path & The location of the source data.\\
Source Primary & The top-level cell within the source data.\\
LVS Report & Where to save the LVS comparison report.\\
\hline
Pex Netlist & The resulting netlist from parasitics extraction.
\end{tabular}
	\end{center}
\end{table}

\section{DRC}
For DRC, the invocation works as follows:

\begin{lstbashplain}
 calibre -drc [options] rule_file
\end{lstbashplain}

\section{LVS}
For LVS, the invocation works as follows:
\begin{lstbashplain}
calibre -lvs [options] rule_file
\end{lstbashplain}

\section{PEX}
Parasitics extraction is a three-step process in itself. These three steps will
be explained next:

\begin{enumerate}
	\item Create a \acrfull{PHDB}\\
		The Calibre xRC User's Manual \citep{calibre2014xrc} states:
		\begin{quote}
The Persistent Hierarchical Database, usually referred to as the PHDB, contains information
about your design’s layout, connectivity, and devices necessary for calculating the parasitic
information.
		\end{quote}
	\begin{lstbashplain}
calibre -xrc -phdb SVRF_file
	\end{lstbashplain}
	\item Create the \acrfull{PDB}\\
		The Calibre xRC User's Manual \citep{calibre2014xrc} states:
		\begin{quote}
Once you have created the PHDB, you create the PDB. The PDB stores the parasitic models for
each extracted net.
This step can be run multiple times without regenerating the PHDB. You might want to do this
if you are extracting different types of parasitics on different nets or if you are also extracting
inductance with the Calibre ® xL Parasitic Inductance Engine.
		\end{quote}
	\begin{lstbashplain}
calibre -xrc -phb [-r|-c|-rc|-rcc] SVRF_file
	\end{lstbashplain}
	\item Create a Netlist of Report\\
		The Calibre xRC User's Manual \citep{calibre2014xrc} states:
		\begin{quote}
As the last step, you produce a netlist or report using the Calibre xRC formatter. The netlist can
be in any of several formats such as HSPICE or DSPF. You can also set the formatter to
perform different types of reductions to produce netlists that are more easily simulated.
		\end{quote}
	\begin{lstbashplain}
calibre -xrc -fmt SVRF_file
	\end{lstbashplain}
\end{enumerate}

Detailed documentation of the tools that are explained in this chapter, can be
found via the following links (that should work in the KIP).
\begin{itemize}
	\item
		\href{file:///cad/products/mentor/calibre_2014.2_33.25/docs_cal_2014.2_14.13/docs/pdfdocs/svrf_ur.pdf}{Standard
		Verification Rule Format (SVRF) Manual}
	\item
		\href{file:///cad/products/mentor/calibre_2014.2_33.25/docs_cal_2014.2_14.13/docs/pdfdocs/calibre_ver_user.pdf}{
		Calibre ® Verification User’s Manual}
	\item \href{file:///cad/products/mentor/calibre_2014.2_33.25/docs_cal_2014.2_14.13/docs/pdfdocs/xrc_user.pdf}
		{Calibre ® xRC TM User’s Manual}
\end{itemize}
