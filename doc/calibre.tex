\chapter{Checking and Extracting Layouts with Calibre}
\label{chap:calibre}

\ti{Calibre} is a tool by \ti{Mentor Graphics} that allows to run several
checks on \gls{ASIC} design files. These checks include:
\begin{itemize}
	\item \acrfull{DRC}
		checks a layout design file for violations of the design rules that are
		specified by the manufacturer.
	\item \acrfull{LVS}
		checks a layout design file for correctness according to a so called
		source netlist (usually a schematic) of the circuit that has been drawn
		in this layout.
	\item \acrfull{PEX}
		extracts the parasitic passive elements from a circuit's layout and
		generates a \ti{Spice} netlist which contains the explicitly present
		elements of the circuit plus the parasitic passive elements.
\end{itemize}

All of these features can be invoked via the same tool, called \ti{Calibre}.
Furthermore, all of these tools use so called \acrfull{SVRF} files to specify
most of the options for the checks and to specify the input files.
\Cref{tab:svrf} shows the \gls{SVRF} rules that are used to specify the inputs
and outputs for Calibre's tasks.

\begin{table}[h]
	\begin{center}
\begin{tabular}{l|l}
Layout System & The format of the layout data.\\
Layout Path & The location of the layout data.\\
Layout Primary & The top-level cell within the layout data.\\
DRC Results Database & The results database pathname and format.\\
\hline
Layout System & The format of the layout data.\\
Layout Path & The location of the layout data.\\
Layout Primary & The top-level cell within the layout data.\\
Source System & The format of the source data.\\
Source Path & The location of the source data.\\
Source Primary & The top-level cell within the source data.\\
LVS Report & Where to save the LVS comparison report.\\
\hline
Layout System & The format of the layout data.\\
Layout Path & The location of the layout data.\\
Layout Primary & The top-level cell within the layout data.\\
Pex Netlist & The resulting netlist from parasitics extraction.
\end{tabular}
\caption{\gls{SVRF} file commands to specify inputs and outputs of a
\ti{Calibre} task. Left column shows the commands, right column contains a
description. The first block of rows can be used for \gls{DRC}, second block
for \gls{LVS} and the third block for \gls{PEX}.}
\label{tab:svrf}
	\end{center}
\end{table}

\ti{BrICk} handles the generation of the rule file for each task automatically.
The user only has to specify a \ti{GDSII} file and a top-level cell name (in
the case of \gls{DRC} and \gls{PEX}) and a source \gls{spice} netlist (in the
case of \gls{LVS} and possibly \gls{PEX}). For examples of how to set up
\ti{grICk} to use \ti{Calibre}, please refer to \cref{sec:calibre_recipes}.

A technology-dependent rule file will be necessary for all of the checks that
can be run with \ti{Calibre}. This rule file will usually be provided by the
manufacturer in the correct \gls{SVRF} format that can be understood by
\ti{Calibre}.

\section{DRC}
The \gls{DRC} only needs a layout file and a technology-dependent rule file as
inputs. It will only check the geometries in the layout against the rules that
are given by the manufacturer. These rules can, for example, constrain the
density of a metal layer (by a lower or upper bound) or they can define minimum
distances between geometries on any layer.

To invoke \ti{Calibre} for DRC, the command looks as follows:

\begin{lstbashplain}
 calibre -drc [options] rule_file
\end{lstbashplain}

\ti{BrICk} will automatically set up this call with the correct options for
you.

\section{LVS}
The \gls{LVS} tool needs as an input a layout file, a netlist file and a
technology-dependent rule file. It will extract the geometries from the layout
and, using the definitions in the rule file, combine them to recognize devices,
such as transistors, resistors and capacitors for example.
The tool will then compare the extracted layout netlist (since the connectivity
between the devices can also be extracted from the layout) with the so called
source netlist, that will usually be derived from a schematic. Therefore, this
tool is called \acrlong{LVS}.

To invoke \ti{Calibre} for \gls{LVS}, the command looks as follows:
\begin{lstbashplain}
calibre -lvs [options] rule_file
\end{lstbashplain}

\ti{BrICk} will automatically set up this call with the correct options for
you.

\subsection{Hierarchical LVS}
For detailed information about hierarchical \gls{LVS}, see also chapter 14 in
\citep{calibre2014ver}.
The Calibre Verification User's Manual \citep{calibre2014ver} states:
\begin{quote}
Calibre nmLVS-H is a hierarchical LVS application. Calibre nmLVS-H maintains
the database hierarchy and exploits this hierarchy to reduce processing time,
memory use, and LVS discrepancy counts.
\end{quote}

Hierarchically corresponding cells have to be explicitly defined in a so called
\ti{hcell file}. This file contains one correspondence per line where
corresponding layout and source cell name are separated by a space in each
line.

With \ti{brICk}, you can specify the content of the \ti{hcell file} as a
\ti{Python} list via the hcell property of the task generator that defines the
\gls{LVS}. An example is given in \cref{sec:calibre_recipes_pex}.

About the question of how many cells should be specified in the hcell list, the
Calibre Verification User's Manual \citep{calibre2014ver} says:
\begin{quote}
In general, an hcell list should be a relatively brief list of corresponding
cell pairs, such that these cells are placed numerous times in the hierarchy.
Logic cells or RAM array cells are good examples of cells to specify in an
hcell list.

Providing an exhaustive list of all cells in the hierarchy is not needed and is
often counter-productive for performance. For example, via cells should
generally not be in the hcell list. This is because circuit extraction can
selectively flatten the hierarchy to improve performance in some cases.
Specifying certain cells like via cells or other very small cells could impede
the performance-improvement heuristics in Calibre nmLVS-H. Omitting a cell from
the hcell list does not necessarily mean it is entirely expanded because it may
have hcells in its hierarchy that are preserved.

If it is the intent to preserve hierarchy regardless of the performance cost,
then using a comprehensive hcell list serves that purpose.
\end{quote}

Another important feature of \ti{Calibre's} \gls{LVS} is the ability to compare
some cells only to the level of their port connections and to not check the
layout inside these cells. This method is called black-boxing. The mechanism to
specify cells that should be treated as black boxes is the \gls{SVRF} rule
\tf{LVS BOX}. The following example shows how to use it. For more information
see page 938 in \citep{calibre2014svrf}.
\begin{verbatim}
LVS BOX LAYOUT A_lay
LVS BOX SOURCE A_src
\end{verbatim}

To introduce such additional statements into the rule file that is generated by
\ti{brICk}, please use the \tf{mixins} attribute for the task generator of
\gls{LVS}. See also \cref{sec:calibre_recipes_pex}.

\section{PEX}
Parasitics extraction is a three-step process in itself. These three steps will
be explained next:

\begin{enumerate}
	\item Create a \acrfull{PHDB}\\
		The Calibre xRC User's Manual \citep{calibre2014xrc} states:
		\begin{quote}
The Persistent Hierarchical Database, usually referred to as the PHDB, contains
information about your design’s layout, connectivity, and devices necessary for
calculating the parasitic information.
		\end{quote}
		This database can be generate by running \ti{Calibre} with the
		following command:
	\begin{lstbashplain}
calibre -xrc -phdb SVRF_file
	\end{lstbashplain}
	However, if you want the parasitics netlist and the parasitics report to
	contain the same device and net names as your schematic, you can run
	\gls{LVS} instead of this step. The \gls{PHDB} will then be generated by
	\gls{LVS} and contain the correct source names.
	\item Create the \acrfull{PDB}\\
		The Calibre xRC User's Manual \citep{calibre2014xrc} states:
		\begin{quote}
Once you have created the PHDB, you create the PDB. The PDB stores the parasitic models for
each extracted net.
This step can be run multiple times without regenerating the PHDB. You might want to do this
if you are extracting different types of parasitics on different nets or if you are also extracting
inductance with the Calibre ® xL Parasitic Inductance Engine.
		\end{quote}
		This step actually extracts the parasitics from the layout. It can be
		invoked via the following command:
	\begin{lstbashplain}
calibre -xrc -phb [-r|-c|-rc|-rcc] SVRF_file
	\end{lstbashplain}

		To specify the amount of detail that will be extracted you can use the
		following command-line options:
		\begin{itemize}
			\item resistance (-r)
			 \item lumped capacitance (-c)
			 \item resistance and distributed capacitance (-rc)
			 \item resistance with distributed capacitance and coupled capacitance between nets (-rcc)
		\end{itemize}

		The Calibre xRC User's Manual \citep{calibre2014xrc} states:
		\begin{quote}
		 There is a trade-off between the amount of detail and how long the
		 netlist takes to simulate. For example, a netlist with parasitics as
		 lumped capacitance takes less time to simulate than one with coupled
		 capacitance between nets, which takes less time than one
		 with coupled capacitance between nets including floating nets.
		\end{quote}

		You can also distinguish between a flat extraction, which flattens all
		design hierarchy and extracts parasisitcs for everything not
		explicitly excluded, and full hierarchical extraction, which extracts
		each cell listed in an \ti{xcell file} only once.

		To make use of this feature you need to run \gls{LVS} with an hcell
		file that contains the cells that you want to extract hierarchically.
		The lines of the xcell file contain a layout cell name, an optional
		source cell name and a flag.
		For example:
		\begin{verbatim}
// layout_name source_name flag
NOR  NOR  -I     //treated as an ideal cell
NAND NAND -P     //treated as a primitive
INV  INV         //handling depends on the extraction type
NMOS NMOS -PCDEF //treated as a pcell
		\end{verbatim}

	Please refer to \cref{sec:calibre_recipes_pex} for a running example.

	\item Create a Netlist of Report\\
		The Calibre xRC User's Manual \citep{calibre2014xrc} states:
		\begin{quote}
As the last step, you produce a netlist or report using the Calibre xRC formatter. The netlist can
be in any of several formats such as HSPICE or DSPF. You can also set the formatter to
perform different types of reductions to produce netlists that are more easily simulated.
		\end{quote}
	\begin{lstbashplain}
calibre -xrc -fmt SVRF_file
	\end{lstbashplain}
\end{enumerate}

Detailed documentation of the tools that are explained in this chapter, can be
found via the following links (that should work in the KIP).
\begin{itemize}
	\item
		\href{file:///cad/products/mentor/calibre_2014.2_33.25/docs_cal_2014.2_14.13/docs/pdfdocs/svrf_ur.pdf}{Standard
		Verification Rule Format (SVRF) Manual}
	\item
		\href{file:///cad/products/mentor/calibre_2014.2_33.25/docs_cal_2014.2_14.13/docs/pdfdocs/calbr_ver_user.pdf}{
		Calibre ® Verification User’s Manual}
	\item \href{file:///cad/products/mentor/calibre_2014.2_33.25/docs_cal_2014.2_14.13/docs/pdfdocs/xrc_user.pdf}
		{Calibre ® xRC TM User’s Manual}
\end{itemize}

\section{Inspecting Errors}
The \ti{Calibre} software suite also comes with a GDSII viewer and a browser for check results.
This program is called \tf{calibredrv}. It can be invoked the following way to view a GDSII file:

	\begin{lstbashplain}
calibredrv -m <filename>
	\end{lstbashplain}

To start it directly with a check result browser you can use one of the following commands:

	\begin{lstbashplain}
calibredrv -m <filename> -rve -drc <drc_result_file>
calibredrv -m <filename> \
           -rve -lvs <lvs_database_svdb> <top-level cell name>
	\end{lstbashplain}

\ti{BrICk} can set this up for you. See \cref{sec:calibre_recipes_drc} for an example.
