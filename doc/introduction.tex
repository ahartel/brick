\setchapterpreamble[u]{%
 \dictum[Larry Wall]{Using a simple tool to solve a complex problem does not
     result in a simple solution.}
 \vspace{1cm}}
\chapter{Introduction}

The \ti{brick} project is a set of python scripts intended to facilitate the
development of mixed-signal \Glspl{ASIC}. It was developed at the
Electronic Vision(s) Group of Heidelberg University.

\ti{brick} is based on the \ti{Waf} build system. \ti{Waf} is best explained by
citing the Waf book \cite{wafbook}:
\begin{displayquote}
	The Waf framework is somewhat different from traditional build systems in
	the sense that it does not provide support for a specific language. Rather,
	the focus is to support the major usecases encountered when working on a
	software project. As such, it is essentially a library of components that
	are suitable for use in a build system, with an emphasis on extensibility.
	Although the default distribution contains various plugins for several
	programming languages and different tools (c, d, ocaml, java, etc), it is
	by no means a frozen product. Creating new extensions is both a standard
	and a recommended practice.
\end{displayquote}

Encouraged by this description of \ti{Waf}, several extensions to it have been
created. These extensions are described in \cref{chap:bricktools}.
For a detailed introduction to \ti{Waf}, the interested reader is referred to
the online version of the \ti{Waf} book
\cite{wafbook}\footnote{\url{https://waf.io/book/}}. A short introduction to
\ti{Waf} is given in \cref{chap:waf}.

\Cref{chap:incisive} describes the simulation tools that come with Cadence
Incisive and explains why it makes sense to simplify some tasks by introducing
\ti{Waf} extensions. \Cref{chap:modelsim} does the same for the Modeltech
simulation software package.

Methods for \gls{LVS}, \gls{DRC} and \gls{PEX} with the \ti{Calibre} tools by
\ti{Mentor} are shortly outlined in \cref{chap:calibre}.

\Cref{chap:recipes} finally gives some example \tf{wscript}s that show how to use the \ti{Waf} tools that are described in this document.

\section{Installation}
\label{sec:install}

On the machines of the \ti{ASIC lab}, you can simply load the latest brick
module. To find out which versions are available, you can enter
\mint{bash}|module av brick|
Once you've found the version number, enter
\mint{bash}|module load brick/<version>|

To install \ti{brICk} on your own machine, the following steps are necessary:
\begin{enumerate}
	\item Clone the repository from
		\mint{bash}|git@brainscales-r@kip.uni-heidelberg.de:brick.git|
	\item Add to your PYTHONPATH:
		\begin{lstbashplain}
export PYTHONPATH=$PYTHONPATH:/your_brick_path/source/waf:\
/your_brick_path/source/python
		\end{lstbashplain}
	\item Add an environment variable called \tf{BRICK\_DIR}:
		\begin{lstbashplain}
export BRICK_DIR=/your_brick_path/
		\end{lstbashplain}
\end{enumerate}

